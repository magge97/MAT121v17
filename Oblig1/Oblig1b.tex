\documentclass[a4paper, norsk, 12pt]{extarticle}
\usepackage[T1]{fontenc} % Vise norske tegn
\usepackage[latin1]{inputenc} % For � kunne skrive norske tegn
\usepackage{babel} % Tilpasning til norsk
\usepackage{graphicx} % For � inkludere grafikk
\usepackage{amsmath,amssymb} % Ekstra matematikkfunksjoner
\usepackage{titlesec}

% Juster sidemarginer
\usepackage[margin=.9in, includefoot]{geometry}

% \usepackage{adjustbox}
% \usepackage{tabularx}
% \usepackage{relsize}
% \usepackage{booktabs}

% Topptekst
\usepackage{fancyhdr}
\pagestyle{fancy}
\fancyhead[L]{Oblig1b - 2017}
\fancyhead[C]{MAT121}
\fancyhead[R]{Magnus �ian}
\usepackage{pdfpages}


% Redefinerer "\section" til "Oppgave n"
% Nummerering skjer automatisk
\titleformat{\section}{\center \normalfont \Large \bfseries}
{Oppgave\ \thesection}{2.3ex plus .2ex}{}
\titlespacing{\subsection}{2em}{*1}{*5}

\begin{document}

%Oppgave 1
\section{}
L�sningene av likningssystemet er gitt ved den augmenterte matrisen:
\begin{align} \label{eqn: oppg1}
		\begin{bmatrix}  1 & 2 & 2 & 1 \\  0 & 1 & 1 & 1 \\ 0 & 1 & -2 & -1 \\ 2 & 2 & a & 0  \end{bmatrix}
	\sim  \begin{bmatrix}  1 & 2 & 2 & 1 \\  0 & 1 & 1 & 1 \\ 0 & 0 & -3 & -2 \\ 0 & -2 & a-4 & -2  \end{bmatrix}
	\sim  \begin{bmatrix}  1 & 0 & 0 & -1 \\  0 & 1 & 0 & 1/3 \\ 0 & 0 & 1 & 2/3 \\ 0 & 0 & 0 & 2(2-a)/3  \end{bmatrix}
\end{align}

\begin{itemize}

%1a)
\item[a)]
Ved � la $a \neq 2$ f�r vi en rad p� formen $ \begin{bmatrix} 0 & 0 & 0 & b \end{bmatrix}$ hvor $ b \neq 0 $.
Alts� har likningssystemet ingen l�sning.

\item[b)]
Systemet har ingen frie variabler.
Ved � la $a = 2$ har vi ikke en rekke p� formen $ \begin{bmatrix} 0 & 0 & 0 & b \end{bmatrix} $ med $b \neq 0$, og f�lgelig har likningssystemet kun �n l�sning.

\item[c)]
Ingen verdier av $a$ gj�r at likningssystemet har uendelig mange l�sninger ettersom det ikke har frie variabler.

\item[d)]
Fra \eqref{eqn: oppg1} ser vi for $a=2$ at likningssystemet har l�sningene $x_1 = -1$, $x_2 = 1/3$ og $x_3 = 2/3$.

\end{itemize}


% Oppgave 2
\section{}
\begin{itemize}

% 2a)
\item[a)]
$\mathbf{b} \in \text{Span} \{\mathbf{v_1},...,\mathbf{v_n}\}$ dersom $\mathbf{b}$ er en line�r kombinasjon av $\{\mathbf{v_1},...,\mathbf{v_n}\}$.

% 2b)
\item[b)]
\begin{align*}
    3\mathbf{v_1} &= 3 \cdot \begin{bmatrix}  1 \\  0 \\ 1  \end{bmatrix} = \begin{bmatrix} 3 \\ 0 \\ 3  \end{bmatrix}
         \text{  og  }
    (-1)\mathbf{v_1} = - \begin{bmatrix}  1 \\  0 \\ 1  \end{bmatrix}  = \begin{bmatrix} -1 \\ 0 \\ -1 \end{bmatrix}
\end{align*}

% 2c)
\item[c)]
\begin{align*}
    0 \cdot \mathbf{v_1} + 0 \cdot \mathbf{v_2} &= \mathbf{0} \text{, } \quad
    0 \cdot \mathbf{v_1} + 1 \cdot \mathbf{v_2} = \begin{bmatrix}  0 \\  0 \\ 0  \end{bmatrix} + \begin{bmatrix}  0 \\  2 \\ -1  \end{bmatrix}  = \begin{bmatrix}  0 \\  2 \\ -1  \end{bmatrix} \text{,  } \\
    1 \cdot \mathbf{v_1} + 1 \cdot \mathbf{v_2} &= \begin{bmatrix}  1 \\  0 \\ 1  \end{bmatrix}  + \begin{bmatrix}  0 \\  2 \\ -1  \end{bmatrix} = \begin{bmatrix} 1 \\ 2 \\ 0 \end{bmatrix} \quad \text {og} \quad
    2 \cdot \mathbf{v_1} + 2 \cdot \mathbf{v_2} = \begin{bmatrix}  2 \\  0 \\ 2  \end{bmatrix}  + \begin{bmatrix}  0 \\  4 \\ -2  \end{bmatrix} = \begin{bmatrix} 2 \\ 4 \\ 0 \end{bmatrix}
\end{align*}

% 2d)
\item[d)]
Eksempel p� vektor $\mathbf{u} \notin \text{Span} \{\mathbf{v_1}, \mathbf{v_2}\}$: $\mathbf{u} = \begin{bmatrix} 1 \\ 1 \\ 1 \end{bmatrix}$. Vi ser av den augmenterte matrisen
\begin{align*}
\begin{bmatrix} 1 & 0 & 1 \\ 0 & 2 & 1 \\ 1 & -1 & 1 \end{bmatrix} \sim \begin{bmatrix} 1 & 0 & 0 \\ 0 & 1 & 0 \\ 0 & 0 & 1 \end{bmatrix}
\end{align*}
at det ikke finnes vekter $c_1$ og $c_2$ slik at $c_1 \mathbf{v_1} + c_2 \mathbf{v_2} = \mathbf{u} $
ettersom vi har en rad p� formen $\begin{bmatrix} 0 & 0 & b \end{bmatrix} $ med $b \neq 0$.
\end{itemize}

% Oppgave 3
\section{}

\begin{itemize}

% 3a)
\item[a)]
Vektorene $\{\mathbf{v_1}, ..., \mathbf{v_p}\}$ er line�rt uavhengige dersom ingen av vektorene er en multippel av noen av de andre vektorene.

% 3b)
\item[b)]
Minst �n av vektorene $\{\mathbf{v_1}, ..., \mathbf{v_p}\}$ er en multippel av de andre vektorene.

% 3c)
\item[c)]
Eksempel p� $4$ uavhengige vektorer i $\mathbb{R}^4$:
\begin{align*}
\begin{bmatrix}  1 \\  0 \\ 0 \\ 0  \end{bmatrix},
\begin{bmatrix}  0 \\  1 \\ 0 \\ 0  \end{bmatrix},
\begin{bmatrix}  0 \\  0 \\ 1 \\ 0  \end{bmatrix}
\text{ og }
\begin{bmatrix}  0 \\  0 \\ 0 \\ 1  \end{bmatrix}.
\end{align*}

Vektorene har verdi ulik null i ulike rader og vil dermed ikke kunne uttrykkes som en multippel av de andre vektorene (vektorene til identitetsmatrisen $I_4$).
\end{itemize}


% Oppgave 4
\section{}
\begin{itemize}
%4a)
\item[a)]
Fra-mengde: $\mathbb{R}^3$, til-mengde: $\mathbb{R}^2$.

%4b)
\item[b)]
La $\boldsymbol{u} = \begin{bmatrix} u_1 \\ u_2 \\ u_3 \end{bmatrix}$ og $\boldsymbol{v} = \begin{bmatrix} u_1 \\ u_2 \\ u_3 \end{bmatrix}$ v�re vektorer i $\mathbb{R}^3$ og $c$ og $d$ skalarer. Da har vi at

% T(cu + dv)
\begin{align*}
T(c \boldsymbol{u} + d \boldsymbol{v})
= T\left(c \begin{bmatrix} u_1 \\ u_2 \\ u_3 \end{bmatrix} + d \begin{bmatrix} v_1 \\ v_2 \\ v_3 \end{bmatrix}\right)
&= T\left(\begin{bmatrix} cu_1 + dv_1 \\ cu_2 + dv_2 \\ cu_3 + dv_3 \end{bmatrix} \right) \\�
&= \begin{bmatrix} 3(cu_1 + dv_1) + 2(cu_2 + dv_2)  \\ -(cu_2 + dv_2) + cu_3 + dv_3\end{bmatrix}
\end{align*}
% cT(u) + dT(v)
\begin{align*}
cT( \boldsymbol{u}) +  d T(\boldsymbol{v}) = c T\left(\begin{bmatrix} u_1 \\ u_2 \\ u_3 \end{bmatrix} \right) + d T\left(\begin{bmatrix} v_1 \\ v_2 \\ v_3 \end{bmatrix}\right)
= c \begin{bmatrix} 3u_1 + 2u_2 \\ -u_2 + u_3 \end{bmatrix} + d \begin{bmatrix} 3v_1 + 2v_2 \\ -v_2 + v_3 \end{bmatrix} \\�= \begin{bmatrix} 3(cu_1 + dv_1) + 2(cu_2 + dv_2)  \\ -(cu_2 + dv_2) + cu_3 + dv_3\end{bmatrix}
\end{align*}
Vi har vist at $T(c \boldsymbol{u} + d \boldsymbol{v}) = cT( \boldsymbol{u}) +  d T(\boldsymbol{v})$. Transformasjonen $T$ er dermed per definisjon line�r.

%4c)
\item[c)]
Vi har at
\begin{align*}
\begin{bmatrix} x_1 \\ x_2 \\ x_3 \end{bmatrix}
= x_1 \begin{bmatrix} 1 \\0 \\ 0 \end{bmatrix} + x_2 \begin{bmatrix} 0\\1\\0 \end{bmatrix} + x_3 \begin{bmatrix} 0 \\0 \\ 1 \end{bmatrix}.
\end{align*}
La
\begin{align*}
\boldsymbol{\ell_1} = \begin{bmatrix} 1 \\0 \\ 0 \end{bmatrix}, \boldsymbol{\ell_2} \begin{bmatrix} 0\\1\\0 \end{bmatrix} \text{�og } \boldsymbol{\ell_3} = \begin{bmatrix} 0 \\0 \\ 1 \end{bmatrix}.
\end{align*}
Det gir at
\begin{align*}
T\left( \begin{bmatrix} x_1 \\ x_2 \\ x_3 \end{bmatrix} \right)
&= T\left(x_1 \boldsymbol{\ell_1} + x_2 \boldsymbol{\ell_2} + x_3 \boldsymbol{\ell_3}\right)
= x_1T\left(\boldsymbol{\ell_1} \right) + x_2 T\left(\boldsymbol{\ell_2} \right) + x_3 T\left(\boldsymbol{\ell_3} \right) \\
&= x_1\begin{bmatrix} 3 \\ 0 \end{bmatrix} + x_2 \begin{bmatrix} 2 \\ -1 \end{bmatrix} + x_3 \begin{bmatrix} 0 \\ 1 \end{bmatrix}
= \begin{bmatrix} 3 & 2 & 0 \\ 0 & -1 & 1 \end{bmatrix} \begin{bmatrix} x_1 \\ x_2 \\ x_3 \end{bmatrix}.
\end{align*}
Standardmatrisen til $T$ er alts� gitt ved:
\begin{align} \label{eqn: standard}
\begin{bmatrix} 3 & 2 & 0 \\ 0 & -1 & 1 \end{bmatrix}.
\end{align}

%4d)
\item[d)]
La $A$ v�re matrisen \eqref{eqn: standard} og $\boldsymbol{b}$ vektor i $\mathbb{R}^2$. Reduserer den augmenterte matrisen $\begin{bmatrix} A & \boldsymbol{b} \end{bmatrix}$:
\begin{align} \label{eqn: redusert}
	\begin{bmatrix} 3 & 2 & 0 & b_1\\ 0 & -1 & 1 & b_2 \end{bmatrix}
\sim \begin{bmatrix} 3 & 0 & 2 & b_1 + 2b_2 \\ 0 & 1 & -1 & - b_2 \end{bmatrix}
\sim \begin{bmatrix} 1 & 0 & 2/3 & 1/3(b_1 + 2b_2) \\ 0 & 1 & -1 & - b_2 \end{bmatrix}.
\end{align}
Vi ser at \eqref{eqn: redusert} har �n fri variabel og ingen rad p� formen $\begin{bmatrix} 0 & 0 & 0 & c\end{bmatrix}$ med $c \neq 0$. Det gir at:
\begin{itemize}
\item Likningen $T(\boldsymbol{x}) = \boldsymbol{b}$ har l�sning for hver $\boldsymbol{b} \in \mathbb{R}^2$, og $T$ er dermed p�.
\item Likningen $T(\boldsymbol{x}) = \boldsymbol{0}$ har flere enn den trivielle l�sningen $\boldsymbol{x} = \boldsymbol{0}$, og $T$ er dermed ikke �n-til-�n.
\end{itemize}

\end{itemize}
\end{document}
